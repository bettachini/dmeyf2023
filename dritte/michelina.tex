\documentclass[aspectratio=169]{beamer} % proyector 16:9 | Ref:  https://en.wikibooks.org/wiki/LaTeX/Presentations
\usetheme{Antibes}
\beamertemplatenavigationsymbolsempty %% Sin barra navegación
\usecolortheme{beaver}


%% Spanska!
\usepackage[utf8]{inputenc}
\usepackage{lmodern} % no producia letras en matemático | Ref: https://tex.stackexchange.com/questions/250413/error-when-using-greek-symbol-in-subscript-in-beamer-presentation 
\usepackage[spanish]{babel}
\def\spanishoptions{argentina}

%% inclusión de gráficas
\usepackage{graphicx}	% instalar ghostscript-x para que el dvi muestre los eps
\graphicspath{ {./graphs/} {../figuras/} }
\usepackage{rotating}	% epígrafe rotado

\begin{document}

\section{Modelo}

\begin{frame}
  \frametitle{Modelo de attition para la campaña de retención proactiva de clientes}
  \begin{columns}[onlytextwidth]
    \begin{column}{.6\textwidth}
      \begin{block}{Objetivo}
        \begin{itemize}
          \item Bajas del \emph{Paquete Premium}, 154 k clientes
          \item Predición dos meses previos (BAJA+2)
		    \end{itemize}
      \end{block}
 
      \begin{block}{Estimación del modelo finalizado}
        \begin{itemize}
          \item \# Envíos de ofertas de retención = 9.5k 
			    \item Ganancia = 125.4 M AR\$
		    \end{itemize}
      \end{block}
    \end{column}
  \end{columns}
\end{frame}



\begin{frame}
  \frametitle{Algoritmo}
  \begin{columns}[onlytextwidth]
    \begin{column}{.6\textwidth}
      \begin{block}{}
        \begin{itemize}
			    \item LightGBM {\tiny (lightgbm\_3.3.5) }% en lenguaje R(R6\_2.5.1)
			    \item Entrenamiento y ensayo contra foto 202107\\
			    $\rightarrow$ reutilización código z824 
			    \item Optimización Bayesiana de hiperparámetros\\
			    $\rightarrow$ z823 
		    \end{itemize}
      \end{block}
    \end{column}
  \end{columns}
\end{frame}


\section{Experimentos}

\begin{frame}
  \frametitle{Insumo}
  \begin{columns}[onlytextwidth]
    \begin{column}{.6\textwidth}
      \begin{block}{Datos crudos}
       \begin{itemize}
			    \item 152 atributos de 154k clientes
          \item fotos 201901 a 202109% =  $\approx$ 5M registros
		    \end{itemize}
      \end{block}

      \begin{block}{Baseline $\Rightarrow$ 456 atributos}
        \begin{itemize}
          \item lag 1, 3 y 6 meses 
          \item Extensión JupySQL operando sobre\\base SQL DuckDB {\tiny (duckdb\_0.2.6)}
		    \end{itemize}
        {\tiny sql\_eng\_baseline.ipynb}
      \end{block}

   \end{column}
  \end{columns}
\end{frame}


\subsection{Feature engineering histórico}

\begin{frame}
  \frametitle{}
  \begin{columns}[onlytextwidth]
    \begin{column}{.6\textwidth}
     \begin{block}{FEH 1 = baseline + $\Rightarrow$ 912 atributos}
        \begin{itemize}
          \item lags 2, 4 y 5 meses $\rightarrow$ semestre
          \item min, máx, media movil semestre
          \item drop reiterados
          \item atm\_oth $>$ atm\_oth
          \item tarjeta = visa + mc (m/c)
          \item proxy actividad = sum[abs(m*)]
		    \end{itemize}
        {\tiny sql\_eng\_all6cat.ipynb}
      \end{block}
     
      \begin{block}{FEH 2 = FEH 1 + $\Rightarrow$ 2128 atributos}
        \begin{itemize}
          \item normalización semestral: atr / media semestral
          \item $\Delta$ normalizado (atr - lag\#)/ media semestral
		    \end{itemize}
        {\tiny sql\_eng\_all6.ipynb}
      \end{block}
    \end{column}
  \end{columns}
\end{frame}


\begin{frame}
  \frametitle{Rendimientos FEH 1 y FEH 2}
  \begin{columns}[onlytextwidth]
    \begin{column}{.6\textwidth}
      \begin{block}{Ganancia vs \# Envíos}
        \includegraphics[width = \columnwidth]{gananciaVsEnvíos.png}
      \end{block}
    \end{column}
  \end{columns}      
\end{frame}


\begin{frame}
  \frametitle{Distribución de ganancias}
  \begin{columns}[onlytextwidth]
    \begin{column}{.6\textwidth}
      \begin{block}{Envíos con máxima ganancia en FEH2}
        \includegraphics[width = \columnwidth]{distribuciónGanancia.png}
      \end{block}
    \end{column}
  \end{columns}      
\end{frame}



\subsection{Imputación de datos}

\begin{frame}
  \frametitle{Catástrofes}
  \begin{columns}[onlytextwidth]
    \begin{column}{.6\textwidth}
      \begin{block}{Análisis de catástrofes}
        \begin{itemize}
          \item Atributos montos (m*) [flotantes]
          \item o cantidad (c*) [enteros]
          \item \textgreater 10 k registros == 0.0 / 0 en foto
          \item $\Rightarrow$ NULL todos los registros en foto
        \end{itemize}
        {\tiny sql\_eng\_all6cat.ipynb}
      \end{block}
    \end{column}
  \end{columns}      
\end{frame}



\subsection{Semillerío}

\begin{frame}
  \frametitle{sale}
  \begin{columns}[onlytextwidth]
    \begin{column}{.6\textwidth}
        {\tiny dritte824\_1129.r}\\
        {\tiny 1129\_null\_sem.r}
    \end{column}
  \end{columns}      
\end{frame}






\end{document}